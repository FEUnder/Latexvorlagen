\documentclass{scrarticle}
\usepackage[
%showframe,
left=2cm,
right=2cm,
top=2cm,
bottom=2cm
]{geometry} %Layout des Dokuments
\usepackage{hyperref}

\usepackage{listings}
\usepackage{color}
\definecolor{lst-gray}{rgb}{0.98,0.98,0.98}
\definecolor{lst-blue}{RGB}{40,0.0,255}
\definecolor{lst-green}{RGB}{65,128,95}
\definecolor{lst-red}{RGB}{200,0,85}
\lstset{
	commentstyle=\color{lst-green},
	basicstyle=\small\ttfamily,
	backgroundcolor=\color{lst-gray},
	breaklines=true,
	captionpos=b,
	columns=fixed,
	extendedchars=true,
	escapeinside={(*}{*)}, 
	frame=single,
	framesep=2pt,
	keepspaces=true,
	keywordstyle=\color{lst-blue},
	language={TeX},
	numbers=left,
	numberstyle=\small\ttfamily,
	showstringspaces=false,
	stringstyle=\color{lst-red},
	tabsize=2,
}
\begin{document}
	\section{Code}
	LaTeX (TeXstudio) ist in der Lage, gewisse Code-Blöcke zu erstellen, die man im Anhang einer Hausarbeit ... anfügen sollte. Hierbei ist zu erwähnen, dass es ab einer gewissen Größe und Komplexität ein Problem hat, dieses zu importieren $\rightarrow$ Google hilft meistens da aus.
	\\\\\noindent
	Für die Einstellung sollte folgendes Übernommen werden:
	\lstinputlisting[language=TeX]{Einstellungen.txt}
	\vspace*{1cm}\noindent
	Es gibt zusätzlich viele weitere Informationen, da hilft ebenso Google - Hier aber vorab schon eine hilf-reiche Seite:
	\href{https://en.wikibooks.org/wiki/LaTeX/Source\_Code\_Listings}{LateX-Hilfe für Code-Einbindung (https://en.wikibooks.org/wiki/LaTeX/Source\_Code\_Listings)}
	
	\section{Einbindung durch externe Datei}
	\lstinputlisting[language=TeX]{lstlisting-example.txt} %Einbindung durch eine externe Datei (ist meistens hilfreich für die Übersicht)
	
	\section{Externe Datei und gewählte Zeilen}
	\lstinputlisting[language=TeX, firstline=2, lastline=4]{lstlisting-example.txt} %Einbindung durch eine externe Datei (ist meistens hilfreich für die Übersicht)
	
	\newpage\noindent
	\section{Ohne externe Datei}
	\begin{lstlisting}[language=TeX]
		\section{Image mit Label / Referenz}
		\begin{figure}[h!hbt]
			\includegraphics[width=\linewidth]{./Abbildungen/Beispiel.jpg}
			\label{Abb: Beispiel1} %Einzigartig pro Arbeit! Dadurch kann man eine Direkt-Referenz auf das Bild erstellen.
		\end{figure}
	\end{lstlisting}
	
\end{document}
