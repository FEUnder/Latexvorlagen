\documentclass{scrarticle}
\usepackage[
%showframe,
left=2cm,
right=2cm,
top=2cm,
bottom=2cm
]{geometry} %Layout des Dokuments
\usepackage{listings}
\begin{document}
	\section{normale Liste}
	\begin{itemize}
		\item Gegenstand 1
		\item Gegenstand 2
	\end{itemize}

	\section{Liste mit verschiedenen Präfixes}
	\begin{itemize}
		\item[\labelitemi] Gegenstand 1
		\item[\labelitemii] Gegenstand 2
		\item[\labelitemiii] Gegenstand 3
		\item[\labelitemiv] Gegenstand 4
		\item[Belieber Präfix] Gegenstand 5
	\end{itemize}
	Dies ist auch möglich, im Vorweg schon zu entscheiden:
	\begin{lstlisting}[language=TeX]
		\renewcommand{\labelitemi}{$\bullet$}
		\renewcommand{\labelitemii}{$\bullet$}
		\renewcommand{\labelitemiii}{$\bullet$}
		\renewcommand{\labelitemiv}{$\bullet$}
	\end{lstlisting}
	, wobei \textbf{\$$\backslash$bullet\$} einem beliebigen Präfix geändert werden kann.
	
	\section{Liste in einer Liste in...}
	\begin{itemize}
		\item Erste Ebene
		 \begin{itemize}
			\item Zweite Ebene
			 \begin{itemize}
				\item Dritte Ebene
				 \begin{itemize}
					\item Vierte Ebene
				\end{itemize}
			\end{itemize}
		\end{itemize}
	\end{itemize}

	\section{Aufzählung}
	\begin{enumerate}
		\item Gegenstand 1
		\item Gegenstand 2
	\end{enumerate}
	Bei der Aufzählung ist alles identisch wie bei einer Liste. Hier sind die Präfixe jedoch voreingestellt. Folgende Präfixe kann man auswählen:
	\begin{itemize}
		\item[] $\backslash$theenumi
		\item[] $\backslash$theenumii
		\item[] $\backslash$theenumiii
		\item[] $\backslash$theenumiv
	\end{itemize}
	Wenn man auch hier im Vorwege das bearbeiten möchte:
	\begin{lstlisting}[language=TeX]
		\renewcommand{\labelenumi}{\theenumi}
		\renewcommand{\labelenumii}{\theenumii}
		\renewcommand{\labelenumiii}{\theenumiii}
		\renewcommand{\labelenumiv}{\theenumiv}
	\end{lstlisting}
	, wobei $\backslash$theenumi-iv die entsprechende Präfixe sind.
\end{document}
