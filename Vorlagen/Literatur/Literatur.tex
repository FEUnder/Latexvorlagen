\documentclass{scrarticle}
\usepackage[
%showframe,
left=2cm,
right=2cm,
top=2cm,
bottom=2cm
]{geometry} %Layout des Dokuments
\usepackage{listings}

\usepackage{hyperref}
\usepackage[ngerman]{babel}
\usepackage[backend=biber,
style=authoryear,
natbib=true,
url= true,
urldate = comp,
dateabbrev = false,
backref = false,
maxcitenames = 2,
sortcites = false
]{biblatex}
\addbibresource{literatur.bib}
\DefineBibliographyStrings{german}{
	andothers = {et al.}}
\DefineBibliographyStrings{ngerman}{%
	urlseen = {Abruf vom}}
\renewcommand*{\nameyeardelim}{\space}

\begin{document}
	\section{Literatur}
	Für die Literatur empfiehlt es sich eine .bib-Datei anzulegen. Hier für gibt es diverse Programme, die man für die Pflege der Quellen verwenden kann. Ich persönlich empfehle \textbf{JabRef}.\\\\\noindent
	Folgender Codeblock wird verwendet, um das Literaturverzeichnis in \textbf{Harvard-Style} zu formatieren. (Angepasst auf meine Bedürfnisse, muss entsprechend überarbeitet werden...):
	\begin{lstlisting}[language=TeX]
		\usepackage{hyperref}
		\usepackage[ngerman]{babel}
		\usepackage[backend=biber,
		style=authoryear,
		natbib=true,
		url= true,
		urldate = comp,
		dateabbrev = false,
		backref = false,
		maxcitenames = 2,
		sortcites = false
		]{biblatex}
		\DefineBibliographyStrings{german}{
			andothers = {et al.}}
		\DefineBibliographyStrings{ngerman}{%
			urlseen = {Abruf vom}}
		\renewcommand*{\nameyeardelim}{\space}
	\end{lstlisting}
	\vspace*{1cm}\noindent
	Um eine Bibliothek (.bib) einzubinden, benötigt man folgenden Code: 
	\begin{lstlisting}[language=TeX]
		\addbibresource{Deine-Bibliothek.bib}
	\end{lstlisting}
	Dies sollte möglichst vor dem $\backslash$DefineBibliographyStrings... passieren.

	\section{Zitieren}
	Hierfür gibt es mehrere Möglichkeiten:
	\begin{itemize}
		\item $\backslash$autocite\{Cite-Key\} $\rightarrow$ Ist für die Harvard-Style (und Formatierung von mir) die beste Auswahl: \autocite{Beispiel}
		\item $\backslash$autocite[vorne stehend][hinten stehend]\{Cite-Key\}: \autocite[vgl.][S.42]{Beispiel}
		\item  $\backslash$cite\{Cite-Key\}: \cite{Beispiel}
		\item $\backslash$cite[vorne stehend][hinten stehend]\{Cite-Key\}: \cite[vgl.][S.42]{Beispiel}
	\end{itemize}
	
	\newpage
	
	\section{Literaturverzeichnis}
	%Diesen Code am Ende hinzufügen, und es sollte passen.
	\setcounter{biburllcpenalty}{9000}% Kleinbuchstaben
	\setcounter{biburlucpenalty}{9000}% Großbuchstaben
	\printbibliography[title=Literaturverzeichnis]
\end{document}
