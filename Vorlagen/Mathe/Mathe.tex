\documentclass{scrarticle}
\usepackage[
%showframe,
left=2cm,
right=2cm,
top=2cm,
bottom=2cm
]{geometry} %Layout des Dokuments

\usepackage{amsmath}

\begin{document}
	\Large
	\section{Bruch}
	$\frac{X}{Y}$ \\
	$\frac{\text{Zähler}}{\text{Nenner}}$
	\section{Summe}
	$\sum$
	\section{Multiplikation}
	$\cdot$ \\
	$\coprod$
	\section{Integral}
	$\int_{untere Zahl}^{obere Zahl}$
	\section{Tiefgestellte Zahl}
	$Zahl_{tiefgestellte Zahl}$
	\\\\\textit{Hinweis: Dies gilt auch für Symbole, bspw. Summen-Zeichen!}
	\section{Hochgestellte Zahl}
	$Zahl^{hochgestellte Zahl}$
	\\\\\textit{Hinweis: Dies gilt auch für Symbole, bspw. Summen-Zeichen!}
	\section{Hochgestellte und Tiefgestellte Zahl}
	$Zahl^{hochgestellte Zahl}_{tiefgestellte Zahl}$
	\\\\\textit{Hinweis: Dies gilt auch für Symbole, bspw. Summen-Zeichen!}
	
	\newpage
	
	\section{Mathematische Umgebungen}
	\begin{equation} \label{eq1}
		\begin{split}
			 X 	&= 1 + 2\\
				&= 1 + 1 + 1 \\
				&= 3
		\end{split}
	\end{equation}

	\begin{multline*}
		f(x) = 12x + 36x^2 + 76x^3 + 112x^4 + 224x^5\\
		 + 554x^6 + 129x^7 + 1337x^8 + 66x^9
	\end{multline*}

	\begin{align}
		 X &= 1 + 2 	& Y  &= 1 - 2\\
		   &= 1 + 1 + 1 &	 &= 1 - 1 - 1\\
		   &= 3 		&	 &= -1\\
		 X &= 3 		& Y  &= -1  
	\end{align}
	\newcommand\numberthis{\addtocounter{equation}{1}\tag{\theequation}} %Hilfreicher Command, um die Nummerierung beizubehalten
	\begin{align*}
		X	&= 1 + 2 		& Y &= 1 - 2\\
			&= 1 + 1 + 1 	& 	&= 1 - 1 - 1\\
			&= 3 			& 	&= -1\\
		X	&= 3 			& Y &= -1 \numberthis
	\end{align*}

	\begin{align}
		X	&= 1 + 2 		& Y &= 1 - 2\nonumber \\
		&= 1 + 1 + 1 	& 	&= 1 - 1 - 1\nonumber \\
		&= 3 			& 	&= -1		\nonumber \\
		X	&= 3 			& Y &= -1	
	\end{align}

	\newpage

	\section{Sonderzeichen (seltenere Fälle)}
	$\approx$ \\
	$\sim$ \\
	$\neq$ \\
	$\rightarrow$ \\
	$\leftarrow$
	
	\vspace*{1cm}
	\noindent
	$\bigcap$ \\
	$\bigvee$ \\
	$\bigcup$ \\
	$\bigwedge$ \\
	$\bigotimes$ \\
	$\bigoplus$
	
	\vspace*{1cm}
	\noindent
	
	$\pi$
	\section{Einheiten}
	$\mu$ %Mikro

\end{document}
