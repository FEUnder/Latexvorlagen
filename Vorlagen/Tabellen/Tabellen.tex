\documentclass{scrarticle}
\usepackage[
%showframe,
left=2cm,
right=2cm,
top=2cm,
bottom=2cm
]{geometry} %Layout des Dokuments

\usepackage[table]{xcolor} %Farbe für Tabelle
\begin{document}
	\section{Tabelle ohne Linien}
	\begin{tabular}[hbt]{lcr} %left, center, right
		Spalte 1.1 & Spalte 2 & Spalte 3 \\
		Spalte 1.2 & Spalte 2.2 & Spalte 3.2\\
		Spalte 1.3 & Spalte 2.3 & Spalte 3.3 \\
	\end{tabular}

	\section{Tabelle mit Zwischenlinien}
	\begin{tabular}[hbt]{l|c|r} %left, center, right
		Spalte 1.1 & Spalte 2 & Spalte 3 \\ \hline
		Spalte 1.2 & Spalte 2.2 & Spalte 3.3\\ \hline
		Spalte 1.3 & Spalte 2.3 & Spalte 3.3 \\
	\end{tabular}

	\section{Multicolumn}
	\begin{tabular}[hbt]{|c|c|c|c|r|}
		\hline
		\multicolumn{3}{|c|}{Spalte 1.1} & Spalte 2 & Spalte 3 \\ \hline
		Spalte 1.2 & Spalte 2.2 & Spalte 3.2 & Spalte 4.2 & Spalte 5.2\\ \hline
		Spalte 1.3 & Spalte 2.3 & Spalte 3.3 & Spalte 4.3 & Spalte 5.3\\ \hline
	\end{tabular}

	\section{Größe einer Tabelle ändern}
	\begin{tabular*}{\textwidth}[hbt]{p{5cm}|p{2cm}|p{5cm}}
		Spalte 1.1 & Spalte 2 & Spalte 3 \\ \hline
		Spalte 1.2 & Spalte 2.2 & Spalte 3.3\\ \hline
		Spalte 1.3 & Spalte 2.3 & Spalte 3.3 \\
	\end{tabular*}

	\section{Eine vertikale Linie hinzufügen}
	\begin{tabular}[hbt]{|l|c|r|} %left, center, right
		\hline
		Spalte 1.1\vline Spalte 2 & Spalte 3 & Spalte 4\\ \hline
		Spalte 1.2 & Spalte 2.2 \vline Spalte 3.2 & Spalte 4.2\\ \hline
		Spalte 1.3 & Spalte 2.3 & Spalte 3.3 \vline Spalte 4.3\\ \hline
	\end{tabular}

	\section{Beliebige Größe einer Spalte oder Reihe}
	\begin{tabular}[hbt]{l|c|r} %left, center, right
		{\rule[-12mm]{0cm}{15mm}} %Erzeugt ein schwarze Box [lift]{Breite}{Höhe}
		Spalte 1.1 \hspace{5cm} & Spalte 2 & Spalte 3 \\ \hline
		Spalte 1.2 & Spalte 2.2 & Spalte 3.3\\ \hline
		Spalte 1.3 & Spalte 2.3 & Spalte 3.3 \\
	\end{tabular}
	\newpage
	\section{Tabelle mit Überschrift}
	\begin{table}[hbt]
		\centering
<<<<<<< Updated upstream
		\caption{Beschreibung}
=======
		\caption{Überschrift / Tabellenbeschreibung}
>>>>>>> Stashed changes
		\begin{tabular*}{\textwidth}{|p{5.24cm}|p{5.24cm}|p{5.24cm}|}
			\hline
			Spalte 1.1 & Spalte 2 & Spalte 3 \\ \hline
			Spalte 1.2 & Spalte 2.2 & Spalte 3.3\\ \hline
			Spalte 1.3 & Spalte 2.3 & Spalte 3.3 \\ \hline
		\end{tabular*}
	\end{table}

	\section{Tabelle mit Farben} %Aufpassen, je nach PDF Viewer sieht es schlecht aus bzw. buggy!
	{\rowcolors{2}{green!80!yellow!50}{green!70!yellow!40}
	\begin{tabular*}{\textwidth}{|p{5.24cm}|p{5.24cm}|p{5.24cm}|}
		\hline
		Spalte 1.1 & Spalte 2 & Spalte 3 \\ \hline
		Spalte 1.2 & Spalte 2.2 & Spalte 3.3\\ \hline
		Spalte 1.3 & Spalte 2.3 & Spalte 3.3 \\ \hline
		Spalte 1.4 & Spalte 2.4 & Spalte 3.4 \\ \hline
		Spalte 1.5 & Spalte 2.5 & Spalte 3.5 \\ \hline
	\end{tabular*}
	}
\end{document}
